% Options for packages loaded elsewhere
\PassOptionsToPackage{unicode}{hyperref}
\PassOptionsToPackage{hyphens}{url}
\PassOptionsToPackage{dvipsnames,svgnames,x11names}{xcolor}
%
\documentclass[
  letterpaper,
  DIV=11,
  numbers=noendperiod]{scrartcl}

\usepackage{amsmath,amssymb}
\usepackage{iftex}
\ifPDFTeX
  \usepackage[T1]{fontenc}
  \usepackage[utf8]{inputenc}
  \usepackage{textcomp} % provide euro and other symbols
\else % if luatex or xetex
  \usepackage{unicode-math}
  \defaultfontfeatures{Scale=MatchLowercase}
  \defaultfontfeatures[\rmfamily]{Ligatures=TeX,Scale=1}
\fi
\usepackage{lmodern}
\ifPDFTeX\else  
    % xetex/luatex font selection
\fi
% Use upquote if available, for straight quotes in verbatim environments
\IfFileExists{upquote.sty}{\usepackage{upquote}}{}
\IfFileExists{microtype.sty}{% use microtype if available
  \usepackage[]{microtype}
  \UseMicrotypeSet[protrusion]{basicmath} % disable protrusion for tt fonts
}{}
\makeatletter
\@ifundefined{KOMAClassName}{% if non-KOMA class
  \IfFileExists{parskip.sty}{%
    \usepackage{parskip}
  }{% else
    \setlength{\parindent}{0pt}
    \setlength{\parskip}{6pt plus 2pt minus 1pt}}
}{% if KOMA class
  \KOMAoptions{parskip=half}}
\makeatother
\usepackage{xcolor}
\setlength{\emergencystretch}{3em} % prevent overfull lines
\setcounter{secnumdepth}{-\maxdimen} % remove section numbering
% Make \paragraph and \subparagraph free-standing
\makeatletter
\ifx\paragraph\undefined\else
  \let\oldparagraph\paragraph
  \renewcommand{\paragraph}{
    \@ifstar
      \xxxParagraphStar
      \xxxParagraphNoStar
  }
  \newcommand{\xxxParagraphStar}[1]{\oldparagraph*{#1}\mbox{}}
  \newcommand{\xxxParagraphNoStar}[1]{\oldparagraph{#1}\mbox{}}
\fi
\ifx\subparagraph\undefined\else
  \let\oldsubparagraph\subparagraph
  \renewcommand{\subparagraph}{
    \@ifstar
      \xxxSubParagraphStar
      \xxxSubParagraphNoStar
  }
  \newcommand{\xxxSubParagraphStar}[1]{\oldsubparagraph*{#1}\mbox{}}
  \newcommand{\xxxSubParagraphNoStar}[1]{\oldsubparagraph{#1}\mbox{}}
\fi
\makeatother

\usepackage{color}
\usepackage{fancyvrb}
\newcommand{\VerbBar}{|}
\newcommand{\VERB}{\Verb[commandchars=\\\{\}]}
\DefineVerbatimEnvironment{Highlighting}{Verbatim}{commandchars=\\\{\}}
% Add ',fontsize=\small' for more characters per line
\usepackage{framed}
\definecolor{shadecolor}{RGB}{241,243,245}
\newenvironment{Shaded}{\begin{snugshade}}{\end{snugshade}}
\newcommand{\AlertTok}[1]{\textcolor[rgb]{0.68,0.00,0.00}{#1}}
\newcommand{\AnnotationTok}[1]{\textcolor[rgb]{0.37,0.37,0.37}{#1}}
\newcommand{\AttributeTok}[1]{\textcolor[rgb]{0.40,0.45,0.13}{#1}}
\newcommand{\BaseNTok}[1]{\textcolor[rgb]{0.68,0.00,0.00}{#1}}
\newcommand{\BuiltInTok}[1]{\textcolor[rgb]{0.00,0.23,0.31}{#1}}
\newcommand{\CharTok}[1]{\textcolor[rgb]{0.13,0.47,0.30}{#1}}
\newcommand{\CommentTok}[1]{\textcolor[rgb]{0.37,0.37,0.37}{#1}}
\newcommand{\CommentVarTok}[1]{\textcolor[rgb]{0.37,0.37,0.37}{\textit{#1}}}
\newcommand{\ConstantTok}[1]{\textcolor[rgb]{0.56,0.35,0.01}{#1}}
\newcommand{\ControlFlowTok}[1]{\textcolor[rgb]{0.00,0.23,0.31}{\textbf{#1}}}
\newcommand{\DataTypeTok}[1]{\textcolor[rgb]{0.68,0.00,0.00}{#1}}
\newcommand{\DecValTok}[1]{\textcolor[rgb]{0.68,0.00,0.00}{#1}}
\newcommand{\DocumentationTok}[1]{\textcolor[rgb]{0.37,0.37,0.37}{\textit{#1}}}
\newcommand{\ErrorTok}[1]{\textcolor[rgb]{0.68,0.00,0.00}{#1}}
\newcommand{\ExtensionTok}[1]{\textcolor[rgb]{0.00,0.23,0.31}{#1}}
\newcommand{\FloatTok}[1]{\textcolor[rgb]{0.68,0.00,0.00}{#1}}
\newcommand{\FunctionTok}[1]{\textcolor[rgb]{0.28,0.35,0.67}{#1}}
\newcommand{\ImportTok}[1]{\textcolor[rgb]{0.00,0.46,0.62}{#1}}
\newcommand{\InformationTok}[1]{\textcolor[rgb]{0.37,0.37,0.37}{#1}}
\newcommand{\KeywordTok}[1]{\textcolor[rgb]{0.00,0.23,0.31}{\textbf{#1}}}
\newcommand{\NormalTok}[1]{\textcolor[rgb]{0.00,0.23,0.31}{#1}}
\newcommand{\OperatorTok}[1]{\textcolor[rgb]{0.37,0.37,0.37}{#1}}
\newcommand{\OtherTok}[1]{\textcolor[rgb]{0.00,0.23,0.31}{#1}}
\newcommand{\PreprocessorTok}[1]{\textcolor[rgb]{0.68,0.00,0.00}{#1}}
\newcommand{\RegionMarkerTok}[1]{\textcolor[rgb]{0.00,0.23,0.31}{#1}}
\newcommand{\SpecialCharTok}[1]{\textcolor[rgb]{0.37,0.37,0.37}{#1}}
\newcommand{\SpecialStringTok}[1]{\textcolor[rgb]{0.13,0.47,0.30}{#1}}
\newcommand{\StringTok}[1]{\textcolor[rgb]{0.13,0.47,0.30}{#1}}
\newcommand{\VariableTok}[1]{\textcolor[rgb]{0.07,0.07,0.07}{#1}}
\newcommand{\VerbatimStringTok}[1]{\textcolor[rgb]{0.13,0.47,0.30}{#1}}
\newcommand{\WarningTok}[1]{\textcolor[rgb]{0.37,0.37,0.37}{\textit{#1}}}

\providecommand{\tightlist}{%
  \setlength{\itemsep}{0pt}\setlength{\parskip}{0pt}}\usepackage{longtable,booktabs,array}
\usepackage{calc} % for calculating minipage widths
% Correct order of tables after \paragraph or \subparagraph
\usepackage{etoolbox}
\makeatletter
\patchcmd\longtable{\par}{\if@noskipsec\mbox{}\fi\par}{}{}
\makeatother
% Allow footnotes in longtable head/foot
\IfFileExists{footnotehyper.sty}{\usepackage{footnotehyper}}{\usepackage{footnote}}
\makesavenoteenv{longtable}
\usepackage{graphicx}
\makeatletter
\def\maxwidth{\ifdim\Gin@nat@width>\linewidth\linewidth\else\Gin@nat@width\fi}
\def\maxheight{\ifdim\Gin@nat@height>\textheight\textheight\else\Gin@nat@height\fi}
\makeatother
% Scale images if necessary, so that they will not overflow the page
% margins by default, and it is still possible to overwrite the defaults
% using explicit options in \includegraphics[width, height, ...]{}
\setkeys{Gin}{width=\maxwidth,height=\maxheight,keepaspectratio}
% Set default figure placement to htbp
\makeatletter
\def\fps@figure{htbp}
\makeatother

\usepackage{fontspec}
\usepackage{multirow}
\usepackage{multicol}
\usepackage{colortbl}
\usepackage{hhline}
\newlength\Oldarrayrulewidth
\newlength\Oldtabcolsep
\usepackage{longtable}
\usepackage{array}
\usepackage{hyperref}
\usepackage{float}
\usepackage{wrapfig}
\KOMAoption{captions}{tableheading}
\makeatletter
\@ifpackageloaded{caption}{}{\usepackage{caption}}
\AtBeginDocument{%
\ifdefined\contentsname
  \renewcommand*\contentsname{Table of contents}
\else
  \newcommand\contentsname{Table of contents}
\fi
\ifdefined\listfigurename
  \renewcommand*\listfigurename{List of Figures}
\else
  \newcommand\listfigurename{List of Figures}
\fi
\ifdefined\listtablename
  \renewcommand*\listtablename{List of Tables}
\else
  \newcommand\listtablename{List of Tables}
\fi
\ifdefined\figurename
  \renewcommand*\figurename{Figure}
\else
  \newcommand\figurename{Figure}
\fi
\ifdefined\tablename
  \renewcommand*\tablename{Table}
\else
  \newcommand\tablename{Table}
\fi
}
\@ifpackageloaded{float}{}{\usepackage{float}}
\floatstyle{ruled}
\@ifundefined{c@chapter}{\newfloat{codelisting}{h}{lop}}{\newfloat{codelisting}{h}{lop}[chapter]}
\floatname{codelisting}{Listing}
\newcommand*\listoflistings{\listof{codelisting}{List of Listings}}
\makeatother
\makeatletter
\makeatother
\makeatletter
\@ifpackageloaded{caption}{}{\usepackage{caption}}
\@ifpackageloaded{subcaption}{}{\usepackage{subcaption}}
\makeatother

\ifLuaTeX
  \usepackage{selnolig}  % disable illegal ligatures
\fi
\usepackage{bookmark}

\IfFileExists{xurl.sty}{\usepackage{xurl}}{} % add URL line breaks if available
\urlstyle{same} % disable monospaced font for URLs
\hypersetup{
  pdftitle={Untitled},
  colorlinks=true,
  linkcolor={blue},
  filecolor={Maroon},
  citecolor={Blue},
  urlcolor={Blue},
  pdfcreator={LaTeX via pandoc}}


\title{Untitled}
\author{}
\date{}

\begin{document}
\maketitle


\subsection{Homework 3}\label{homework-3}

\section{Part 1. Setup}\label{part-1.-setup}

\begin{Shaded}
\begin{Highlighting}[]
\CommentTok{\# read in libraries}
\FunctionTok{library}\NormalTok{(tidyverse)}
\end{Highlighting}
\end{Shaded}

\begin{verbatim}
-- Attaching core tidyverse packages ------------------------ tidyverse 2.0.0 --
v dplyr     1.1.4     v readr     2.1.5
v forcats   1.0.0     v stringr   1.5.1
v ggplot2   3.5.1     v tibble    3.2.1
v lubridate 1.9.4     v tidyr     1.3.1
v purrr     1.0.4     
-- Conflicts ------------------------------------------ tidyverse_conflicts() --
x dplyr::filter() masks stats::filter()
x dplyr::lag()    masks stats::lag()
i Use the conflicted package (<http://conflicted.r-lib.org/>) to force all conflicts to become errors
\end{verbatim}

\begin{Shaded}
\begin{Highlighting}[]
\FunctionTok{library}\NormalTok{(here)}
\end{Highlighting}
\end{Shaded}

\begin{verbatim}
here() starts at /Users/tanveersingh/github/ENVS-193DS_homework-03
\end{verbatim}

\begin{Shaded}
\begin{Highlighting}[]
\FunctionTok{library}\NormalTok{(flextable)}
\end{Highlighting}
\end{Shaded}

\begin{verbatim}

Attaching package: 'flextable'

The following object is masked from 'package:purrr':

    compose
\end{verbatim}

\begin{Shaded}
\begin{Highlighting}[]
\FunctionTok{library}\NormalTok{(janitor)}
\end{Highlighting}
\end{Shaded}

\begin{verbatim}

Attaching package: 'janitor'

The following objects are masked from 'package:stats':

    chisq.test, fisher.test
\end{verbatim}

\begin{Shaded}
\begin{Highlighting}[]
\FunctionTok{library}\NormalTok{(readxl)}

\CommentTok{\# read in personal data}
\NormalTok{mydata }\OtherTok{\textless{}{-}} \FunctionTok{read\_csv}\NormalTok{(}\StringTok{"Personal Data Project {-} Sheet1.csv"}\NormalTok{)}
\end{Highlighting}
\end{Shaded}

\begin{verbatim}
Rows: 13 Columns: 9
-- Column specification --------------------------------------------------------
Delimiter: ","
chr  (4): Date, Time of day, Location, Light
dbl  (4): Number of pages, Total time of reading, Distractions #, Comprehension
time (1): Time started

i Use `spec()` to retrieve the full column specification for this data.
i Specify the column types or set `show_col_types = FALSE` to quiet this message.
\end{verbatim}

\section{Part 2. Problems}\label{part-2.-problems}

\subsection{a.}\label{a.}

To summarize the data and compare a response variable between
categories, I could calculate the pages per minute for each session to
compare reading effectiveness across different locations. This
comparison would be informative because different environments might
offer varying levels of comfort or light, which could impact my focus,
and consequently, how well I understood the material.

\subsection{b.}\label{b.}

\begin{Shaded}
\begin{Highlighting}[]
\NormalTok{df }\OtherTok{\textless{}{-}} \FunctionTok{clean\_names}\NormalTok{(mydata) }\SpecialCharTok{|\textgreater{}} \CommentTok{\# clean column names}
  \CommentTok{\# adding pages\_per\_minute column}
  \FunctionTok{mutate}\NormalTok{(}\AttributeTok{pages\_per\_minute =} \FunctionTok{ifelse}\NormalTok{(total\_time\_of\_reading }\SpecialCharTok{\textgreater{}} \DecValTok{0}\NormalTok{,  number\_of\_pages }\SpecialCharTok{/}\NormalTok{ total\_time\_of\_reading, }\DecValTok{0}\NormalTok{))}

\NormalTok{mean\_pages\_per\_minute }\OtherTok{\textless{}{-}}\NormalTok{ df }\SpecialCharTok{|\textgreater{}} \CommentTok{\# create new data frame for mean pages per minute}
  \FunctionTok{group\_by}\NormalTok{(location) }\SpecialCharTok{|\textgreater{}} \CommentTok{\# group by location}
  \FunctionTok{summarise}\NormalTok{(}\AttributeTok{Mean\_PPM =} \FunctionTok{mean}\NormalTok{(pages\_per\_minute)) }\SpecialCharTok{|\textgreater{}} \CommentTok{\# calculate the mean}
  \FunctionTok{rename}\NormalTok{(}\StringTok{"Location"} \OtherTok{=}\NormalTok{ location) }\CommentTok{\# rename the location to be capitalized}

\FunctionTok{ggplot}\NormalTok{(df, }\CommentTok{\# plotting the data}
       \FunctionTok{aes}\NormalTok{(}\AttributeTok{x =}\NormalTok{ location, }\CommentTok{\# location on x axis}
               \AttributeTok{y =}\NormalTok{ pages\_per\_minute)) }\SpecialCharTok{+} \CommentTok{\# Pages per minute on the y axis}
         \FunctionTok{geom\_boxplot}\NormalTok{(}\FunctionTok{aes}\NormalTok{(}\AttributeTok{fill =}\NormalTok{ location)) }\SpecialCharTok{+} \CommentTok{\# creating boxplot}
  \FunctionTok{geom\_jitter}\NormalTok{(}\AttributeTok{width =} \FloatTok{0.2}\NormalTok{, }\AttributeTok{size =} \DecValTok{1}\NormalTok{, }\AttributeTok{color =} \StringTok{"black"}\NormalTok{, }\AttributeTok{alpha =} \FloatTok{0.7}\NormalTok{) }\SpecialCharTok{+} \CommentTok{\# adding the underlying data}
  \FunctionTok{labs}\NormalTok{( }\CommentTok{\# label function to rename labels}
    \AttributeTok{x =} \StringTok{"Reading Location"}\NormalTok{, }\CommentTok{\# new x axis title}
    \AttributeTok{y =} \StringTok{"Pages Per Minute"}\NormalTok{, }\CommentTok{\# new y axis title}
    \AttributeTok{title =} \StringTok{"Reading Efficiency (Pages Per Minute) by Location"} \CommentTok{\# title for visualization}
\NormalTok{  )}
\end{Highlighting}
\end{Shaded}

\includegraphics{Untitled_files/figure-pdf/unnamed-chunk-2-1.pdf}

\subsection{c.}\label{c.}

Data collected from reading sessions between May 12-26, 2025. Efficiency
measured as pages read per minute

\subsection{d.}\label{d.}

\begin{Shaded}
\begin{Highlighting}[]
\NormalTok{table1 }\OtherTok{\textless{}{-}} \FunctionTok{flextable}\NormalTok{(mean\_pages\_per\_minute) }\SpecialCharTok{|\textgreater{}} \CommentTok{\# create table using flextable}
  \FunctionTok{colformat\_double}\NormalTok{(}\AttributeTok{j =} \StringTok{"Mean\_PPM"}\NormalTok{, }\AttributeTok{digits =} \DecValTok{1}\NormalTok{) }\SpecialCharTok{|\textgreater{}} \CommentTok{\# rounding to nearest decimal}
  \FunctionTok{autofit}\NormalTok{() }\SpecialCharTok{|\textgreater{}} \CommentTok{\# autofit function to make everything fit}
  \FunctionTok{theme\_booktabs}\NormalTok{() }\SpecialCharTok{|\textgreater{}} \CommentTok{\# theme}
  \FunctionTok{align}\NormalTok{(}\AttributeTok{align =} \StringTok{"center"}\NormalTok{, }\AttributeTok{part =} \StringTok{"all"}\NormalTok{) }\CommentTok{\# center align all parts of the table}

\NormalTok{table1 }\CommentTok{\# displaying the table}
\end{Highlighting}
\end{Shaded}

\global\setlength{\Oldarrayrulewidth}{\arrayrulewidth}

\global\setlength{\Oldtabcolsep}{\tabcolsep}

\setlength{\tabcolsep}{2pt}

\renewcommand*{\arraystretch}{1.5}



\providecommand{\ascline}[3]{\noalign{\global\arrayrulewidth #1}\arrayrulecolor[HTML]{#2}\cline{#3}}

\begin{longtable*}[c]{|p{1.13in}|p{1.08in}}



\ascline{1.5pt}{666666}{1-2}

\multicolumn{1}{>{\centering}m{\dimexpr 1.13in+0\tabcolsep}}{\textcolor[HTML]{000000}{\fontsize{11}{11}\selectfont{\global\setmainfont{Helvetica}{Location}}}} & \multicolumn{1}{>{\centering}m{\dimexpr 1.08in+0\tabcolsep}}{\textcolor[HTML]{000000}{\fontsize{11}{11}\selectfont{\global\setmainfont{Helvetica}{Mean\_PPM}}}} \\

\ascline{1.5pt}{666666}{1-2}\endfirsthead 

\ascline{1.5pt}{666666}{1-2}

\multicolumn{1}{>{\centering}m{\dimexpr 1.13in+0\tabcolsep}}{\textcolor[HTML]{000000}{\fontsize{11}{11}\selectfont{\global\setmainfont{Helvetica}{Location}}}} & \multicolumn{1}{>{\centering}m{\dimexpr 1.08in+0\tabcolsep}}{\textcolor[HTML]{000000}{\fontsize{11}{11}\selectfont{\global\setmainfont{Helvetica}{Mean\_PPM}}}} \\

\ascline{1.5pt}{666666}{1-2}\endhead



\multicolumn{1}{>{\centering}m{\dimexpr 1.13in+0\tabcolsep}}{\textcolor[HTML]{000000}{\fontsize{11}{11}\selectfont{\global\setmainfont{Helvetica}{Balcony}}}} & \multicolumn{1}{>{\centering}m{\dimexpr 1.08in+0\tabcolsep}}{\textcolor[HTML]{000000}{\fontsize{11}{11}\selectfont{\global\setmainfont{Helvetica}{0.5}}}} \\





\multicolumn{1}{>{\centering}m{\dimexpr 1.13in+0\tabcolsep}}{\textcolor[HTML]{000000}{\fontsize{11}{11}\selectfont{\global\setmainfont{Helvetica}{Bed}}}} & \multicolumn{1}{>{\centering}m{\dimexpr 1.08in+0\tabcolsep}}{\textcolor[HTML]{000000}{\fontsize{11}{11}\selectfont{\global\setmainfont{Helvetica}{0.4}}}} \\





\multicolumn{1}{>{\centering}m{\dimexpr 1.13in+0\tabcolsep}}{\textcolor[HTML]{000000}{\fontsize{11}{11}\selectfont{\global\setmainfont{Helvetica}{Living\ Room}}}} & \multicolumn{1}{>{\centering}m{\dimexpr 1.08in+0\tabcolsep}}{\textcolor[HTML]{000000}{\fontsize{11}{11}\selectfont{\global\setmainfont{Helvetica}{0.5}}}} \\

\ascline{1.5pt}{666666}{1-2}



\end{longtable*}



\arrayrulecolor[HTML]{000000}

\global\setlength{\arrayrulewidth}{\Oldarrayrulewidth}

\global\setlength{\tabcolsep}{\Oldtabcolsep}

\renewcommand*{\arraystretch}{1}

\section{Problen 2, Affective
visualization}\label{problen-2-affective-visualization}

\subsection{a.}\label{a.-1}

An affective visualization for my personal reading data project can be
visually represented with orbs in a jar, where each orb represents an
individual reading session. The color of the orb can represent the
comprehension: vibrant green colors for high comprehension, fading from
purple to muddy brown hues for low comprehension. The surface of the orb
can represent the distractions, becoming rougher, more jagged as the
number of distractions grows. Also, the total time spent reading can be
represented through the size of the orb.

\subsection{b.}\label{b.-1}

\begin{figure}[H]

{\centering \includegraphics{./Users/tanveersingh/github/ENVS-193DS_homework-03/IMG_0317.png}

}

\caption{Jar of reading orbs}

\end{figure}%




\end{document}
